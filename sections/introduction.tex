\section{Introduction}

Over the last several years, there has been a growing interest in learning
outside of the traditional classroom setting. Whether it is because departments
do not offer classes in a certain area, or because the format of semester-long
classes does not mesh well with the topics covered, there is a demand for new
approaches to teaching the skills and ideas needed to do science. One common way
to teach new skills is a short, time-bounded learning event, such as a bootcamp
or short-course. These attempt to compress several topics into an intensive
learning session that is usually held over one or several days.

These kinds of time-bounded bootcamps offer many advantages for learning over
longer and less-frequent courses. For example, they allow the students to focus
entirely one one topic for an extended period of time. This can be particularly
useful for material that demands a ``deeper dive'' and intensive working. It is
also particularly useful for topics that attempt to jointly cover both
conceptual material and more ``hands-on'' material, because the increased time
leaves more room for experimentation, discussion, and active learning. Finally,
the opt-in nature of these courses often ensures that students are more
motivated to learn, and their proximity both to one another as well as to the
instructors makes for a good learning environment.

A common event of this nature often follows the same rough formula. First,
instructors develop materials on their own computers, perhaps eventually sharing
them together in the same github repository. On the days preceding the event,
instructors send links to github repositories, as well as instructions for how
to download the proper packages and computing environments. On the day of,
instructors assume that students have already figured out how to install this
material on their computers, or often hold mini ``startup'' sessions that assist
students that have problems in getting their environments set up. During the
bootcamp, material is covered that was developed using the instructor computers,
meaning that differences in computing power, memory, etc will cause some
students to have failed kernels or exceptionally slow execution. Finally,
students are then sent home with their newly-acquired knowledge and all course
materials they used during the class. This has the benefit of being immediately
usable outside of class (assuming that course materials were able to run during
class). Course materials generally remain on github, though they have a
relatively high barrier to entry for new students to discover because of the
aforementioned environment setup costs.

These types of courses also offer unique challenges in effectively teaching
students. Because these courses emphasize hands-on learning in which students
perform interact with material on their own, the course experience becomes
heavily-dependent on the ability of each student to get started in the first
place. In addition, asking students to run material on their own hardware limits
and adds variability to the types of computations that can be covered in the
class. Finally, because interacting with course materials requires installing
many pieces of custom software, it is rare for course materials to be
re-utilized by new users after the course has completed.

One solution for handling these challenges is to offload the challenge of
student-specific hardware onto a shared cloud computing platform. This approach
standardizes the experience of each student by allowing them access to a single
online resource for the duration of the class. This article covers a recent
attempt at using advanced cyberinfrastructure in order to hold a day-long
bootcamp in machine learning at the University of California in San Francisco.
We will cover the technological advancements necessary in order to host course
materials online and make them available to students, as well as cover the
challenges faced in implementing this course setup effectively.

\ariel{This is here now just so stuff doesn't break on build:} Including some citation \cite{Lamport:LaTeX}
