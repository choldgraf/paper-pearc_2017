\section{Conclusions}

This project represents a first step towards a flexible and easy way to deploy
computational environments on cloud platforms available through national
advanced cyberinfrastructure (ACI) for the purposes of teaching data analytic
methods to scientists. The purpose of this manuscript is to provide inspiration
and a rough guide for how one might deploy a similar approach towards teaching a
bootcamp-style event. It also aims to lay out a path towards refining this
process to accommodate new research domains and training events, and to make it
more straightforward for instructors to set up course infrastructure without the
need for exceptional technical knowledge. Utilizing cloud-computing
infrastructure has the ability to improve both the teaching and learning
experience in data-heavy fields, and offers new opportunities for giving
researchers a pragmatic, hands-on experience with data that focuses on the
topics covered in the course. As the materials available for instructors
improves, we believe that this approach will increase in efficacy and become a
common tool in the toolkit for modern-day pedagogy.

For instructors interested in hosting a similar event at their institution, the
following steps should be taken:

\begin{enumerate}
\item Contact your local XSEDE Campus Champion
\item Apply for an XSEDE Education Allocation for Jetstream
\item Install Docker for Mac or Docker for Windows on laptop
\item Build docker container based on jupyter/datascience-notebook
\item Test instructor's Jupyter notebooks in Docker on laptop
\item Provision virtual machines (VMs) on Jetstream
\item Deploy custom docker container on Jetstream VMs
\item Connect from your browser to Jupyter notebooks running on Jetstream IP address
\end{enumerate}
